% LaTeX document produced by pod2latex from "rotatelog.pod".
% The followings need be defined in the preamble of this document:
%\def\C++{{\rm C\kern-.05em\raise.3ex\hbox{\footnotesize ++}}}
%\def\underscore{\leavevmode\kern.04em\vbox{\hrule width 0.4em height 0.3pt}}
%\setlength{\parindent}{0pt}

\documentclass{article}

\begin{document}

\section{ROTATELOG}%
\index{ROTATELOG}

\subsection*{NAME}
rotatelog --- rotates files when they reach a certain trigger size.

\subsection*{SYNOPSIS}
rotatelog {\tt [}-cdht{\tt ]}

\subsection*{DESCRIPTION}
This is a program to rotate log files.  It can rotate any file listed in
its configuration file.  The file does not necessarily need to be log
file.  This program imitates the {\tt logrotate} and {\tt newsyslog} programs
found on Linux and FreeBSD respectively.  I originally wrote this to
make life easier in Solaris.

\subsection*{OPTIONS}%
\index{OPTIONS}

\begin{description}

\item[{\bf -c}]%
\index{-c@{\bf -c}}%

Specify the location of the configuration file.  Multiple configuration
files can be separated by commas with no spaces.

\item[{\bf -d}]%
\index{-d@{\bf -d}}%

Turn on debug output.  This is verbose output that shows most of what
{\tt rotatelog} does.

\item[{\bf -h}]%
\index{-h@{\bf -h}}%

Print the help information.

\item[{\bf -t}]%
\index{-t@{\bf -t}}%

Turn on the test flag.  This causes {\tt rotatelog} to run as if all files
need rotation, but disallows the actual rotation.  This is most useful
in combination with the {\bf -d} flag.

\end{description}

\subsection*{ERRORS}%
\index{ERRORS}

This program will exit if it cannot open its configuration {\em file\/}(s) and if
any of the system calls fail.  This program will also exit if the there
are any syntax errors in the actions section.  Any other configuration
file errors are ignored and a warning is issued.  All error messages are
printed to {\tt STDERR}.  This program is meant to be run as a cron job so
error messages such as these will be mailed to the owner of the cron
job.  Many times {\tt rotatelog} is run in the {\tt root} crontab.

\subsection*{FILES}
The {\tt rotatelog} program depends upon its configuration file for
information concerning what files to rotate and when to rotate them.
The {\tt rotatelog.conf} file is located in the {\tt /usr/local/etc} directory
by default.  This location is set in the configuration section of the
source code along with the paths to certain system utility functions.
It is possible to specify a new configuration file via the {\tt -c} command
line option, and it is also possible to specify multiple configuration
files in this manner as well using the form {\tt -c file1,file2,file3}.
Note that configuration files that appear later in a multiple
configuration specification can override settings in previous
configuration files.  When installing this program it will be necessary
to check this section of the source code.  Along with specifying the the
files to rotate there is also the ability to specify ownership and mode
of the files after rotation, the number of rotated files to keep, the
compression used or not used on the rotated files, and an auxiliary
action in the form of a shell command to take place when certain files
are rotated.  The {\tt rotatelog} program is normally run as {\tt root}.  It
is important that the config file is only writable by {\tt root} due to the
ability to execute shell commands!

There are three sections in the configuration file.  They can appear in
any order.  Each section is identified by a keyword.  Configuration
information follows the keyword and extends until the next keyword.
Keywords can come in any order and repeat, but it would be best to
follow a simple format such as:
\begin{verbatim}
 FILES:
\end{verbatim}
\begin{verbatim}
 # File  Trigger  Owner:Group  Mode  Compress  Archive_Limit
 # =========================================================
\end{verbatim}
\begin{verbatim}
 /var/log/wtmp      1M   root:wheel  664  none  4
 /var/log/wtmpx     4M   root:wheel  644  none  4
 /var/log/messages  74K  root:wheel  664  Z     5
 /var/log/maillog   60K  root:wheel  664  gz    7
\end{verbatim}
\begin{verbatim}
 ACTIONS:
\end{verbatim}
\begin{verbatim}
 # Shell Command :file1,file2,file3 ... fileN
 # ==========================================
\end{verbatim}
\begin{verbatim}
 rotate1                             : /var/log/wtmp,/var/log/wtmpx
 kill -HUP `cat /var/run/syslog.pid` : /var/log/messages,/var/log/maillog
\end{verbatim}
\begin{verbatim}
 NOTIFY:
\end{verbatim}
\begin{verbatim}
 # Person to email rotation notification
 # =====================================
\end{verbatim}
\begin{verbatim}
 root@localhost
\end{verbatim}

Comment lines must contain a pound sign at the start of the line.  These
comment lines are ignored along with blank lines and lines containing
only whitespace.  In general there can be no leading whitespace on
configuration lines while there can be any amount of whitespace between
elements.

\subsubsection*{Files}%
\index{Files}

The files section specifies which files are to be rotated, when they are
to be rotated, what their ownership and modes are to be post rotation,
how many rotated copies are to be retained, and the compression used on
the files once rotated.  The beginning of the files section is noted by
the occurrence of the {\tt FILES:} keyword.

This information must adhere to the following syntax:
\begin{verbatim}
 File  Trigger  Owner:Group  Mode  Compress  Archive_Limit
\end{verbatim}

For example:
\begin{verbatim}
 /var/log/wtmp      1M   root:wheel  664  none  4
 /var/log/wtmpx     4M   root:wheel  644  none  4
 /var/log/messages  74K  root:wheel  664  Z     5
 /var/log/maillog   60K  root:wheel  664  gz    7
\end{verbatim}

The fields above can be separated by any amount of whitespace.  The
first field is the full path to the file that requires rotation.  The
second field is the {\tt trigger}.  The {\tt trigger} specifies when rotation
is to occur.  The {\tt trigger} is the maximum size of a file in bytes,
kilobytes, or megabytes.  Bytes can be specified with B or b, kilobytes
with K or k, or megabytes can be specified with M or m.  If a file
listed is found to be larger than the {\tt trigger} it is rotated.  The
{\tt owner:group} field specifies who is to own the file once it is
rotated.  The {\tt mode} is the file permission mode in octal notation of
the rotated file as specified in the {\em chmod\/}(1) manual page.  The mode
does not include the optional first digit of the four digit {\em chmod\/}(1)
octal mode.  The mode is only the simple three digit format shown above.
The {\tt compression} field specifies the type of compression to use on the
rotated file.  A file can be compressed with {\tt gzip} (gz), {\tt compress}
(Z), or not at all (none).  The {\tt archive limit} field specifies how
many copies of a rotated file to keep.  This number is the highest count
in the rotation scheme.  The rotation count starts at 0 so this is
actually one higher then the total number of files kept.  If the
{\tt archive limit} were set to 4 for the file {\tt /var/log/wtmp} the maximum
number of archived {\tt wtmp} files would be 5.  The files would appear in
the {\tt /var/log/} directory as:
\begin{verbatim}
 wtmp    <-- the current wtmp file.
 wtmp.0  <-- the first archive of wtmp.
 .
 .
 .
 wtmp.4  <-- the last archive of wtmp.
\end{verbatim}

If the last file of an archive exists, wtmp.4 in the above example, it
is removed during the rotation and the previous archive file wtmp.3 is
renamed accordingly.  All other files are rotated in the same manner
until the current file is reached and rotated.

\subsubsection*{Actions}%
\index{Actions}

The actions section specifies if any actions are to occur for any of the
rotated files.  The actions section is denoted by the {\tt ACTIONS:}
keyword.  Sometimes a process must be notified that it must close its
file descriptors and open new ones due to something like a file
rotation.  Many processes will handle this action when the receive a
{\tt SIGHUP} signal.  Most of the time a {\tt kill -HUP} on the process will
accomplish this.  One example is {\tt syslogd}.  We can see that {\tt syslogd}
has certain files open all of the time by using {\tt lsof} to inquire about
the status of one of its log files:
\begin{verbatim}
 [rowland@darkstar rowland]$ sudo lsof /var/log/messages
 COMMAND PID USER   FD   TYPE   DEVICE SIZE/OFF NODE NAME
 syslogd  99 root    8w  VREG 4,131076     3119  232 /var/log/messages
\end{verbatim}

It is not enough to just move the current file into the archive and
touch a new one for writing.  The {\tt syslogd} process must be told that a
new file exists and that it is to close the current file descriptor and
open a new one on the same file name (not to keep writing to the file
pointed to by the current file descriptor).  This is what the actions
section does.  The actions section has the following syntax:
\begin{verbatim}
 Shell Command : file1, file2, file3, ... fileN
 - or -
 Shell Command : rotate1, rotate2, ... rotateN
\end{verbatim}

For example:
\begin{verbatim}
 rotate1                             : /var/log/wtwp, /var/log/wtmpx
 - or -
 kill -HUP `cat /var/run/syslog.pid` : rotate1
 - or -
 kill -HUP `cat /var/run/syslog.pid` : /var/log/messages,/var/log/maillog
\end{verbatim}

You are free to add whitespace around the {\tt :} and {\tt ,} characters.
This extra whitespace will be ignored.  When the {\tt /var/log/messages}
file is to be rotated it is first moved to a new name, but not
compressed.  The {\tt syslogd} program will still be writing to this file
even though the name has changed because its file descriptor points to
this location on the filesystem and the file descriptor is still open at
this point.  A new file is touched and its permissions and ownership are
set to the values specified in the files section.  Then the {\tt SIGHUP}
signal is the sent to {\tt syslogd}, and it closes the rotated file and
opens the new file that replaces it.  After this is done it is safe to
compress the old file if compression was specified in the files section.

There is also a special action called {\tt rotate} which allows you to bind
multiple files together for rotation.  All of the files must be present
in the files section so that {\tt rotatelog} knows how to rotate them.
When one file is rotated then the other files bound with it are rotated
immediately.  The format of {\tt rotate} is:
\begin{verbatim}
 rotateN : file1, file2, ... fileN
\end{verbatim}

The {\tt rotate} command must be followed by an integer.  This allows for
more than one binding but only for different groups of files.  You may
not leave the integer off.  The {\tt wtmp} file is a good candidate for
this type of rotation.  When the {\tt wtmp} file is rotated it is a good
idea to also rotate the {\tt wtmpx} file.  A binding of:
\begin{verbatim}
 rotate1 : /var/adm/wtmp, /var/adm/wtmpx
\end{verbatim}

will rotate both files whenever one of them needs rotation.  When
binding files for rotation in this manner, there is only one way to
associate another action (a normal shell command) to the rotation of the
bound files.  Using the {\tt rotate} action indicates that all files are to
be rotated at the same time.  If an action is also to occur, we must be
certain that all of the files have begun rotation before executing that
action.  An action on the group of bound files in the {\tt rotate} group is
specified by using that {\tt rotate} action as the bound file for the
normal shell command action.  The {\tt syslogd} example is perfect for
this.  In the {\tt syslogd} case one would probably want to send {\tt syslogd}
a {\tt SIGHUP} signal once all {\tt syslogd} files have been rotated.  This is
how that is accomplished:
\begin{verbatim}
 rotate2                             : /var/log/messages,/var/log/maillog
 kill -HUP `cat /var/run/syslog.pid` : rotate2
\end{verbatim}

This ensures the following steps occur during file rotation:
\begin{verbatim}
 1.  Begin rotation on a file.
          - Figure out the archive count of the current file.
          - Rotate old logs up one count, removing the last if necessary.
          - Rotate the current file up one count.
          - Touch a new version of the file.
\end{verbatim}
\begin{verbatim}
 2.  Perform any actions on the file.
          - Check rotateN bound files.
                 [if file is in a rotateN group]
                 - Begin any file rotation on the other files in the
                   rotateN bound group.  All other files are put
                   through step #1 at this point.
                 - Perform any normal shell command action associated
                   with this group of rotateN bound files.
                 - Finish rotation of all other files bound in the
                   rotateN group.  This is step #3.
                 - Return.  This causes the original file that triggered
                   this recursion to finish its rotation.
          - Check normal shell command actions.
                 [in this case the file was not in a rotateN group]
                 - Perform the shell command action.
                 - Return.  This moves the file into step #3.
\end{verbatim}
\begin{verbatim}
 3.  Finish rotating a file.
          - Compress the previous log file if the file has been specified
            for compression in the files section.
          - Add the file to the list of files which have been rotated.
\end{verbatim}

This process ensures that any files bound together are in the beginning
of file rotation before any shell commands are executed as a result of
this rotation.  Once the shell command action occurs, if there is one,
all of the files finish their rotation.  Due to this flexibility in file
rotation actions, the following rules apply to the actions section:
\begin{verbatim}
 1.  A file may appear in only one rotateN group.
\end{verbatim}
\begin{verbatim}
 2.  A file may appear in only one normal shell command group.
\end{verbatim}
\begin{verbatim}
 3.  A file may not appear in both a rotateN group and a normal shell
     command action.
\end{verbatim}
\begin{verbatim}
 4.  A normal shell command may be bound explicitly to a file that is
     not in a rotateN group.
\end{verbatim}
\begin{verbatim}
 5.  A normal shell command cannot be bound to a file in a rotateN
     group explicitly.  To bind a shell command to files in a rotateN
     group one must implicitly bind the command to the rotateN action.
\end{verbatim}

If there are no actions to perform when any of the files are rotated
this section may be omitted from the configuration file.  If you rotate
log files written to by {\tt syslogd} then you will most certainly require
one of the examples above with the appropriate path to your
{\tt syslogd.pid} file.  If you leave out this step and rotate {\tt syslogd}
files, {\tt syslogd} will most certainly NOT be your friend.  If you choose
to bind {\tt syslogd} files together in a rotateN group, all of the files
will be rotated when one file is rotated.  If you want to define the
action for each file, only when that file is rotated, do not use the
rotateN form of the action.  It all depends on what you wish to do.

\subsubsection*{Notify}%
\index{Notify}

The notify section specifies who is to receive notification in the event
that any files are rotated.  The notify section is started with the
{\tt NOTIFY:} keyword.  The notification is sent out via email to the full
email address specified in this section.  This is the simplest section.
The email address is simply listed as in the following example:
\begin{verbatim}
 # Person to receive email rotation notification
 # =============================================
\end{verbatim}
\begin{verbatim}
 root@localhost
\end{verbatim}

If there is to be no notification this section may be omitted or a value
of "none" may be used.  There can only be one email address in this
section.

\subsection*{RESTRICTIONS}%
\index{RESTRICTIONS}

There can be no leading whitespace on config file lines.  Someday I may
fix that.  The files must be specified with their full paths in all
sections.  This is true in the actions section where you can list an
action for multiple files.  If there are any syntax errors in the
configuration file those lines will be ignored and a warning will be
printed to {\tt STDERR}.  If this program is run as a cron job this will
result in an email message to the owner of the cron job.  In most cases
this will be {\tt root}.

\subsection*{AUTHOR}
This piece of code was written by Shaun Rowland (rowland@interhack.net)
mostly during the early hours of the morning.  In my world that is
considered "day" while afternoon is considered "night".  Go figure.

Copyright 1999, 2000, 2001 Interhack Corporation.  All rights reserved.

\subsection*{HISTORY}%
\index{HISTORY}

\$Log: rotatelog,v \$
Revision 1.7  2001/05/27 18:21:21  rowland
Added hostname information.

Revision 1.6  2001/05/23 15:10:48  rowland
Removed history stuff that killed pod2latex.

Revision 1.5  2001/05/23 15:04:47  rowland
Fixed a bug where extra whitespace at the end of an ACTIONS line would cause
the file eq check file at the end of do {\em action()\/} to fail.  Doh!  Now we
are good to go.

Revision 1.4  2001/05/19 21:23:32  rowland
Updated the following in logrotate 1.3:
\begin{verbatim}
        * Added the -c and -h command line options.
        * Added the B and b trigger size specification.
        * Changed the release Makefile to create installation directories if
          they do not already exist.
\end{verbatim}

Revision 1.3  2001/04/22 23:01:36  rowland
Updated perldoc and config file to better reflect reality in rotating
syslogd files.  You wouldn not normally put these files in a rotateN group
and then define an action for that group because this would cause all of
the files to be rotated when just one of them is rotated.  You could do
this if you wanted, but it makes a perfect example.  Better examples will
follow later, but the examples here should better reflect reality.

Revision 1.2  2001/04/22 22:45:25  rowland
Added more actions section error checking and cleaned up the default settings.

Revision 1.1.1.1  2001/04/22 21:58:26  rowland
The rotatelog program was designed to rotate files and perform actions on
those files once rotated if desired.  This program began its life as
logrotate.  I changed the name so it would not be confused with the GNU
logrotate program.  This version includes improved code for handling
rotateN bound files and actions on those bound files (in other words it
now works properly).

\end{document}
